\documentclass[a4paper,11pt]{article}

\usepackage[utf8]{inputenc}
\usepackage[T1]{fontenc}
\usepackage[ngerman]{babel}
\usepackage{graphicx}

\title{Bloom-Filter: Implementierung und Auswertung}
\author{Daniel Barber}
\date{\today}

\begin{document}

\maketitle

\section*{1. Idee und Funktionsweise}

Ein Bloom-Filter ist eine probabilistische Datenstruktur, die speichert, ob ein Element
\emph{möglicherweise} enthalten ist oder \emph{sicher nicht}.
Er verwendet dazu ein Bitfeld der Länge $m$ und $k$ Hashfunktionen.
Beim Einfügen eines Elements werden $k$ Bit-Positionen gesetzt.
Beim Abfragen werden die gleichen Positionen überprüft.

\textbf{Vorteile:}
\begin{itemize}
  \item Sehr wenig Speicherverbrauch
  \item Sehr schnelle Abfragen
  \item Einfache Implementierung
\end{itemize}

\textbf{Nachteile:}
\begin{itemize}
  \item Es gibt false positives
  \item Einmal gesetzte Bits können nicht entfernt werden (ohne Counting-Bloom-Filter)
\end{itemize}

\section*{2. Beispiel aus der Praxis}

Ein Beispiel für den Einsatz eines Bloom-Filters ist \textbf{Google Safe Browsing}.
Browser wie Chrome verwenden einen Bloom-Filter, um schnell zu prüfen, ob eine URL
in einer Liste gefährlicher Webseiten enthalten sein könnte.
Die Abfrage ist dadurch extrem schnell und speichereffizient.

Weitere Beispiele:
\begin{itemize}
  \item Datenbanken (Cassandra, LevelDB)
  \item Web-Caches
  \item Peer-to-Peer-Netzwerke
\end{itemize}

\section*{3. Test der Fehlerrate}

Für die Aufgabe wurde eine Wortliste aus \texttt{words.txt} mit $n$ Einträgen verwendet.
Der Bloom-Filter wurde mit der gewünschten Fehlerwahrscheinlichkeit $p = 0.01$ erstellt.
Die Werte $m$ und $k$ wurden nach den Standardformeln berechnet.

Anschließend wurde ein Experiment mit vielen zufälligen Nicht-Wörtern durchgeführt, um die
experimentelle Fehlerrate zu messen.

\begin{center}
  \includegraphics[width=\textwidth]{output.png}
\end{center}

\textbf{Beobachtung:}
Die experimentelle Fehlerrate lag nahe beim erwarteten Wert von $p = 0.01$, was zeigt, dass
der Bloom-Filter korrekt implementiert wurde.

\end{document}
